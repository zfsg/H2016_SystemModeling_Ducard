%begin Packages ------------------------------------------------------
\newlength{\mylength}
\setlength{\mylength}{0.6	 cm}
\usepackage[landscape, top=\mylength, bottom=\mylength, left=\mylength, right=\mylength]{geometry} %pagelayout

\usepackage[ngerman]{babel} %language
\usepackage[utf8]{inputenc} %language
\usepackage{multicol} %multiple columns
\usepackage{multirow} %connect cells of tabular vertically
\usepackage{amssymb} %math-symbols
\usepackage{enumitem} %change spaces between listed items [itemsep = x pt]
\usepackage{graphicx} %include pictures
\usepackage{tabulary} %tables with linebreaks
\usepackage{tabularx}
\usepackage{amsmath} %advanced mathematics
\usepackage{enumitem} %enumerations
\usepackage{siunitx} %use units
\usepackage[dvipsnames]{xcolor} %colors for titles
\usepackage{mathrsfs} % used for mathscript
\usepackage{mathtools} %use of rcases*
\usepackage{listingsutf8}%use of other languages
\usepackage{esint} %special integrals

\usepackage{array}   % for \newcolumntype macro
\newcolumntype{L}{>{$}l<{$}} % math-mode version of "l" column type
\newcolumntype{R}{>{$}r<{$}} % math-mode version of "r" column type
\newcolumntype{C}{>{$}c<{$}} % math-mode version of "c" column type


% Set the color of your style
% Avaiable are: Apricot, Aquamarine, Bittersweet, Black, Blue, blue, BlueGreen, BlueViolet, BrickRed, Brown, BurntOrange, CadetBlue, CarnationPink, Cerulean, CornflowerBlue, Cyan, Dandelion, DarkOrchid, Emerald, ForestGreen, Fuchsia, Goldenrod, Gray, Green, GreenYellow, JungleGreen, Lavender, ... (more at: http://en.wikibooks.org/wiki/LaTeX/Colors)
\def\StyleColor{Gray}

% section color box
\setkomafont{section}{\mysection}

\newcommand{\mysection}[1]{	
	\Large\normalfont\scshape
    \setlength{\fboxsep}{0cm}%already boxed
    \vspace{0ex}
    \colorbox{\StyleColor}{%
        \begin{minipage}{\linewidth}%
            \vspace*{4pt}%Space before
            #1
            \vspace*{4pt}%Space after
        \end{minipage}%
     }
     \vspace{1ex}
}

%subsection color box
\setkomafont{subsection}{\mysubsection}

\newcommand{\mysubsection}[1]{%
    \normalsize\normalfont\scshape%
    \vspace{-2ex}
    \setlength{\fboxsep}{0cm}%already boxed
    \colorbox{\StyleColor!60}{%
        \begin{minipage}{\linewidth}%
            \vspace*{4pt}%Space before
             #1
            \vspace*{4pt}%Space after
        \end{minipage}%
    }
    \vspace{-2ex}
}
    
\setkomafont{subsubsection}{\mysubsection}
\newcommand{\mysubsubsection}[1]{%
    \small\normalfont\scshape%
    \vspace{-2ex}
    \setlength{\fboxsep}{0cm}%already boxed
    \colorbox{\StyleColor!10}{%
        \begin{minipage}{\linewidth}%
            \vspace*{4pt}%Space before
             #1
            \vspace*{4pt}%Space after
        \end{minipage}%
    }
    \vspace{-2ex}
}

%-------------------------------------------------------------------

%begin commands
\newcommand{\finn}{\vspace{1.5ex}}

\newcommand{\fix}{\vspace{-23pt}} %reduce wasted space

\newcommand{\Eofr}{\vec{E}(\vec{r})} 	%Elektrisches Feld

\newcommand{\Bofr}{\vec{B}(\vec{r})}	%Magnetisches Feld

\newcommand{\Hofr}{\vec{H}(\vec{r})}	%Magnetische Erregung

\newcommand{\Fofr}{\vec{F}(\vec{r})}	%Kraftfeld

\newcommand{\Vprod}[2]{\vec{#1}\times\vec{#2}} 	%Vektorprodukt

\newcommand{\dahe}{$\rightarrow$}	%easy arrow outside mathmode

\newcommand{\compaq}{\setlength{\itemsep}{0cm}\setlength{\parskip}{0cm}}%compact itemizes

\newcommand{\mypic}[1]{\includegraphics[width=\linewidth]{#1}} %easy including of pictures within the column

\newcommand{\hfull}{\hfill$|$\hfill} %easy seperation of two equations
%end commands

\newcommand{\longeq}{\hfill$=$\hfill}%easy seperation of an equation

\newcommand{\intinf}[1]{\int_{-\infty}^\infty{#1}}

\newcommand{\expval}[1]{\langle #1\rangle}

\newcommand{\important}[1]{\begin{center}\fbox{#1}\end{center}}

\newcommand{\importname}[2]{\begin{center}\fbox{#2}  #1\end{center}}

\newcommand{\mimportant}[1]{\begin{center}\begin{tabular}{|c|}\hline #1 \\ \hline\end{tabular}\end{center}}

\newcommand{\importable}[1]{\begin{center}\begin{tabular}{|ll|}\hline #1 \\ \hline\end{tabular}\end{center}}

\newcommand{\note}[1]{\footnotesize #1 \normalsize}

\newcommand{\lstfill}{\vspace{4ex}}